%%% 2016年 09月 28日 星期三 16:52:46 CST

\documentclass{scrartcl}

\usepackage{mathtools}
\usepackage[top=3cm,left=1cm,right=1cm,bottom=3cm]{geometry}
\usepackage{bookman}
\usepackage{scrpage2}
\usepackage{CJKutf8}
\usepackage[pdftex]{hyperref}
\usepackage{scrtime}
\usepackage{scrdate}
\usepackage{chemfig}

\pagestyle{scrheadings}

\setheadtopline{2pt}
\setheadsepline{.4pt}
\setfootbotline{2pt}
\setfootsepline[text]{.4pt}

\automark[subsection]{section}


\begin{document}

\begin{CJK}{UTF8}{gbsn}

\ohead{李小丹}
\ihead{一个练习}
\chead{\headmark}
\ofoot{\href{mailto:mymail@dummy.com}{mail}}
\ifoot{Tel:23571113}
\cfoot[]{}

\title{使用\KOMAScript编写中文文档}
\subtitle{一个练习}
\author{李小丹}
\date{2016-09-28}

\begin{titlepage}

\maketitle

\begin{abstract}
	本文使用\LaTeX和\KOMAScript来制作中文文档。
\end{abstract}

\vspace{1cm}

{\small{\tableofcontents}}

\end{titlepage}


%~\\\vspace{\stretch{0.5}}

\section{一个公式}
看看一个简单的数学公式排版怎样。\\

\begin{equation}
	e^{\pi i} + 1 = 0
\end{equation}

\subsection{再来几个公式}
\begin{equation}
	\int_0^1 f(x)\,\mathrm{d}x = \varphi
\end{equation}

\begin{align}
	\lim_{n \to \infty} \left(1 + \frac{1}{n} \right) ^ n & = e\\
	\lim_{n \to \infty} \sum_{i=0}^{n} \frac{x^i}{i!} & = e^x
\end{align}

\section{化学式}
捣鼓点化学。\footnote{
	例子来自
	\href{https://en.wikibooks.org/wiki/LaTeX/Chemical\_Graphics}
	{wikibooks}和
	\href{http://latex-community.org/know-how/434-chemistry-molecules}
	{latex-community}} \\

\begin{center}
	\chemfig{*6(=-=-=-)} \\
	\vspace{0.3cm}
	\chemfig{-\chemabove{N}{\scriptstyle\oplus}(=[1]O)-[7]O^{\ominus}} \\
	\vspace{0.3cm}
	\chemfig{*6(=-=(-(=[2]O)-[::-60]O-[0]O-[::30](=[2]O)-[::-60]*6(=-=-=-))-=-)}
\end{center}

%~\\\vspace{\stretch{0.5}}

\clearpage


\section{来个表}

\begin{tabular}{|c|c|}
	\hline
	a & b\\
	\hline
	c & d\\
	\hline
\end{tabular}

\subsection{中文表}
\begin{tabular}{|c|c|}
	\hline
	人 & 口\\
	\hline
	手 & 上\\
	\hline
	中 & 下\\
	\hline
\end{tabular}

\section{其他}
来点别的。

\subsection{时间}
玩玩时间。\\

\begin{minipage}[h]{0.4\linewidth}
\begin{flushleft}

\begin{verse}
Today is \todaysname. \\
今天是\todaysname. \\
现在是\thistime. \\
\end{verse}

%\begin{quotation}
%Today is \todaysname. \\
%今天是\todaysname. \\
%现在是\thistime. \\
%\end{quotation}

\subsection{some Item}
来几个item。

\minisec{来个普通的}
\begin{itemize}
	\item 第一条
	\item 第二条
		\begin{itemize}
			\item 第1条
			\item 第2条
		\end{itemize}
	\item 第三条
\end{itemize}

\end{flushleft}
\end{minipage}
\begin{minipage}[h]{0.5\linewidth}
\begin{flushright}

\minisec{来个description}
~\\
\begin{description}
	\item[一、] 没错,这个是第一。
	\item[二、] 没错,这个是第二。
\end{description}

\minisec{再来个labeling}
它能自动对齐。\\
\begin{labeling}[~~---~]{\usekomafont{descriptionlabel}myheadings}
	\item[\usekomafont{descriptionlabel}这也太长了吧]
		确实很长啊
	\item[\usekomafont{descriptionlabel}copy]
		这行是我复制上一行.
	\item[\usekomafont{descriptionlabel}vim copy很方便]
		这行同样处理了。
\end{labeling}

\end{flushright}
\end{minipage}


\clearpage\end{CJK}

\end{document}
